\documentclass[a4paper,11pt]{nprp}
%%% Important macros
%
% \PI: the name of the PI
% \proposalNo: the number of the proposal
% \draft: comment out to remove instructions
%
% \Part{xxx}: creates proposal part called xxx

\newcommand{\PI}{}
\newcommand{\proposalNo}{NPRP00X-0000-000000}
\newcommand{\proposalTitle}{Project Title}

\newcommand{\draft}{}  % Remove for final version

\usepackage[T1]{fontenc}
%\usepackage[scaled]{uarial}
\usepackage{helvet}
\renewcommand*\familydefault{\sfdefault}

\usepackage{xcolor}                 
\definecolor{dark}{gray}{.5}
\usepackage{fancyhdr}
\usepackage{ifthen}
\usepackage{titlesec}
\usepackage[dotinlabels]{titletoc}
\usepackage{changepage}
\usepackage{setspace}
\usepackage{array}
\usepackage{multirow}
\usepackage{colortbl}
\usepackage{enumerate}

\usepackage{lipsum}
\usepackage{amsmath}
\usepackage{amsthm}
\usepackage{amssymb}
\usepackage{stmaryrd}
\usepackage{proof}
\usepackage[hyphens,spaces,obeyspaces]{url}
%\usepackage{hyperref}           % if you want links in the pdf

%% For highlighting
\usepackage{soul}
\sethlcolor{yellow}

% Table of contents
\titlecontents{section}[0em]{\smallskip\bfseries} % margin and space after item
{\thecontentslabel.\enspace} %numbered sections
{} %numberless section
{\titlerule*[0.4pc]{.}\contentspage} % dots between title and page number

%% Allows changing from portrait to landspace pages and back
\usepackage[usegeometry]{typearea}
\usepackage{geometry}
\geometry{
  left=0.5in, right=0.5in, top=1in, bottom=1.2in
}
\newcommand*{\useportrait}{%
  \clearpage
  \KOMAoptions{paper=portrait,DIV=current} % switch to portrait
  \newgeometry{ % geometry settings for portrait
    left=0.5in, right=0.5in, top=1in, bottom=1.2in
  }%
  \fancyhfoffset{0pt} % recalculate head and foot width for fancyhdr
}
\newcommand*{\uselandscape}{%
  \clearpage
  \KOMAoptions{paper=landscape,DIV=current} % switch to landscape
  \newgeometry{ % geometry settings for landscape
    left=0.5in, right=0.5in, top=1in, bottom=1.2in
  }%
  \fancyhfoffset{0pt} % recalculate head and foot width for fancyhdr
}

\titleformat*{\section}{\normalsize\bfseries}
\titlespacing*{\section}{0pt}{10pt}{5pt}
\titlespacing*{\subsection}{0.1in}{10pt}{5pt}
\titlespacing*{\subsubsection}{0.2in}{10pt}{5pt}

% Dot after section number and space between section number and title
\makeatletter
\renewcommand{\@seccntformat}[1]{\csname the#1\endcsname.\ }
\makeatother

% Radiobuttons on front page
\usepackage{tikz}
\makeatletter
\newcommand*{\radiobutton}{%
  \@ifstar{\@radiobutton0}{\@radiobutton1}%
}
\newcommand*{\@radiobutton}[1]{%
  \begin{tikzpicture}
    \pgfmathsetlengthmacro\radius{height("X")/2}
    \draw[radius=\radius] circle;
    \ifcase#1 \fill[radius=.6*\radius] circle;\fi
  \end{tikzpicture}%
}
\makeatother

\newcommand{\PartName}{}
\newcommand{\Part}[1]
 {\renewcommand{\PartName}{#1}
  \clearpage%
  \setcounter{section}{0}%
  %\setcounter{page}{1}%
  \noindent%
  \textbf{#1}%
  \vspace{3pt}%
  \noindent}

\newcommand{\instructions}[1]
 {\ifthenelse{\isundefined{\draft}}
   {}
   {{\color{dark} \small #1} \smallskip\par}}


%% Headers and footers
\pagestyle{fancy}
\renewcommand{\footrulewidth}{\headrulewidth}
\lhead{}
\rhead{{\small NPRP-S No.: \proposalNo}}
\lfoot{{\small\textbf{QNRF -- NPRP-S 14\textsuperscript{th} cycle Research Plan Form {\color{red}(Revised February 2021)}}}}
\cfoot{}
\rfoot{{\small\textbf{Page \thepage}}}


%% For tables
\renewcommand{\arraystretch}{1.5}
\newcolumntype{P}[1]{m{\dimexpr #1\textwidth-2\tabcolsep}}
% Vertically centered wrapping columns
\newcolumntype{C}[1]{>{\centering\arraybackslash}p{#1}}
% Vertically and horizontally centered wraping columns
\newcolumntype{M}[1]{>{\centering\arraybackslash}m{#1}}
\definecolor{headercolor}{gray}{.8}
\colorlet{wpcolor}{blue!20}
\colorlet{taskcolor}{red!20}
\newlength{\realwidth}


\usepackage{amsmath}
\usepackage{amsthm}
\usepackage{amssymb}
\usepackage{stmaryrd}
\usepackage{proof}
\usepackage[hyphens,spaces,obeyspaces]{url}
%\usepackage{hyperref}           %must be last

% For highlighting
\usepackage{soul}
\sethlcolor{yellow}

\renewcommand{\contentsname}{Table of Contents}
\newenvironment{OLD}{\color{red}}{\color{black}}

% For tables
\renewcommand{\arraystretch}{1.5}
\newcolumntype{P}[1]{m{\dimexpr #1\textwidth-2\tabcolsep}}
\newcolumntype{C}[1]{>{\centering\arraybackslash}P{#1}}
\definecolor{headercolor}{gray}{.8}
\colorlet{wpcolor}{blue!20}
\colorlet{taskcolor}{red!20}

\begin{document}
\thispagestyle{empty}
\begin{center}

\textbf{QNRF -- NATIONAL PRIORITY RESEARCH PROGRAM - STANDARD}

\vspace{0.86in}

\begin{tabular}{|m{2in}|m{4.6in}|}
\hline
\textbf{NPRP-S ID} & \proposalNo \\
\hline
\textbf{Project title in English} & \proposalTitle \\
\hline
\textbf{Project title in Arabic (optional)} & \\
\hline
\textbf{Type of Application} &
  \begin{itemize}
  \item[\radiobutton*] New
  \item[\radiobutton] Resubmission
  \item[\radiobutton] Renewal
  \end{itemize} \\
\hline
\textbf{Type of Research} &
  \begin{itemize}
  \item[\radiobutton] Basic Research
  \item[\radiobutton*] Applied Research
  \item[\radiobutton] Experimental development / Translational Research
  \end{itemize} \\
\hline
\textbf{Priority Theme} & \\
\hline
\textbf{Submitting Institution} & Carnegie Mellon University in Qatar \\
\hline
\textbf{Lead PI (title, name, position)} &  \\
\hline
\textbf{List of participants (PIs' names, collaborative institutions, PIs' residency)} &
  \vspace{5pt}
  \begin{tabular}{|p{1.3in}|p{1.3in}|p{1.3in}|}
  \hline
  PI name & PI institution & PI residency \\
  \hline
   &  & \\
  \hline
   &  & \\
  \hline
  \end{tabular}
  \vspace{5pt} \\
\hline
\textbf{Co-Funding} &
  \begin{itemize}
  \item[\radiobutton] Yes

  {\small (list co-funders here:)}
  \item[\radiobutton*] No
  \end{itemize} \\
\hline
\textbf{Total funding requested} &
  \begin{tabular}{r|p{1.2in}|c}
  42,424,242 USD & \textbf{Project duration} & 42 months
  \end{tabular} \\
\hline
\end{tabular}
\end{center}


\newpage
%\pagestyletoc
\pagestylebody
\doublespacing
\tableofcontents
\singlespacing

\instructions{Before submitting this document:
  \begin{itemize}
    \item Comment out the line defining the command \texttt{draft} to remove
    the instructions.
    \item This document must not exceed 40 pages excluding cover page, table of
    contents and references, body of text in regular black Arial font size 11,
    single space and the margins as identified in the template (no less than 0.5
    inches).
    \item Make sure that the information in the cover page above is matching the
    information entered on QGrants.
  \end{itemize}

  The sections are MANDATORY (unless marked ``if applicable''). It is recommended
  that proposals follow the sub-sections indicated in the template. However,
  applicants are free to adapt the structure of the sub-sections according to
  the project’s needs. Applications that do not comply with the instructions may
  not be accepted for review.}

\newpage
%\pagestylebody
\Part{Research Plan}

\section{PROPOSAL SUMMARY}
\label{sec:prop-summary}
\instructions{Please describe in up to 6,000 characters the nature of the
  proposed research. This should be in plain English and should cover the
  following items: What are the scientific objectives and innovations of this
  project? What does the project-team plan to do? How will the project-team do it?
  What are the expected outcomes? What will be the expected impact of this
  project?}

\section{REBUTTAL (if applicable)}
\instructions{If this proposal is a resubmission, include here the questions \&
responses to previous peer reviewers' comments and identify the substantive
changes incorporated in the research plan.  In addition, LPIs resubmitting
proposals must point out in the body of this research plan (e.g. in bold type,
line in the margin, underlining, italic, etc.) all revisions and modifications
made in response to the PRs' comments.}

\section{OBJECTIVES AND SIGNIFICANCE}
\label{sec:objectives}
\instructions{(Suggested length: 3 to 5 pages)}

\subsection{SCIENTIFIC OR TECHNICAL OBJECTIVES}
\label{sec:objectives:scien}
\instructions{Describe the overall research project. Explain the main aim or
  hypothesis and the research proposed. Anchor the project in the relevant
  literature. 

  If applicable, please indicate the Technology Readiness Level (TRL)
  that you plan to achieve within the scope of this NPRP project. Please refer to
  the list of Technology Readiness Levels at
  \url{https://www.qnrf.org/en-us/Funding/Research-Programs/National-Priorities-Research-Program-NPRP}.}

\begin{adjustwidth}{0.2in}{0in}
\end{adjustwidth}

\subsection{ADVANCES ON STATE-OF-THE-ART}
\label{sec:objectives:adv}
\instructions{Describe the advance(s) your proposal would provide in knowledge
  and understanding in your field beyond the existing knowledge and/or
  technologies. 

  Explain to what extent your proposal explores creative, original,
  novel, or potentially transformative concepts.  

  If your project specifically
  aims to create innovative technologies or processes, then list the existing
  solutions (Prior Art -- patents and products) closest to your research and how
  your approach is better.  Also, list the specific aspects of your research that
  matter to end-users in this area (the benefits an end- user desires in and
  derives from a solution) and how you expect your proposed outcomes to perform
  better than existing available solutions on these requirements (some examples of
  end-user benefits are: low price, ease-of-use, high accuracy, high specificity,
  cost-efficient, low carbon footprint, etc.).}

\begin{adjustwidth}{0.2in}{0in}
\end{adjustwidth}


\subsection{PRELIMINARY DATA OR STUDIES}
\label{sec:objectives:prel}
\instructions{Provide details on preliminary data related to this project and on
  related research projects (ongoing or previous) by members of the research-team,
  including QNRF projects (if any). Indicate any outputs (patents, publications,
  products, data sets, etc.) obtained from those projects.}

\begin{adjustwidth}{0.2in}{0in}
\end{adjustwidth}

\section{METHODOLOGY AND PROJECT STRUCTURE}
\label{sec:method}

\subsection{METHODOLOGY}
\label{sec:method:method}
\instructions{Clearly describe and explain your overall methodology, methods or
  scientific plan, including any tools to be used. Avoid duplication with the
  section~\ref{sec:method:wps}.}

\begin{adjustwidth}{0.2in}{0in}
\end{adjustwidth}


\subsection{RESEARCH PLAN REQUIREMENTS REGARDING THE SECTIONS ON THE PROTECTION
OF HUMAN SUBJECTS SECTION; AND/OR ON THE VERTEBRATE ANIMAL CARE AND USE.}
\label{sec:method:animal}
\instructions{A Protection of Human Subjects section of the Research Plan is
  required for all applications proposing human subjects research. To assist in
  preparing the section on Protection of Human Subjects, possible scenarios on
  human subject involvement in research are provided in the ``Research Ethics and
  Regulatory Requirements'' document at
  \url{https://www.qnrf.org/en-us/Funding/Research-Programs/National-Priorities-Research-Program-NPRP}.
  All research projects will fall into one of these scenarios.  For all scenarios,
  applicants must provide sufficient information to allow the reviewers to
  determine if the designation of human subjects involvement is appropriate. The
  proposed research must meet all the requirements of applicable Ministry of
  Public Health policies for the protection of human subjects from research risks,
  and data and safety monitoring (when applicable).  An overview of the
  information that must be included in the Research Plan for activities involving
  the care and use of vertebrate animals is provided here
  \url{https://www.qnrf.org/en-us/Funding/Research-Programs/National-Priorities-Research-Program-NPRP}.}

\begin{adjustwidth}{0.2in}{0in}
\end{adjustwidth}


\subsection{PROJECT MANAGEMENT}
\label{sec:method:manag}
\instructions{Indicate the organizational aspects of the project and the methods
  of coordination between the various collaborators (how they will meet and
  communicate).}

\begin{adjustwidth}{0.2in}{0in}
\end{adjustwidth}

\newpage
\subsection{TECHNICAL DESCRIPTION BY WORK PACKAGE}
\label{sec:method:wps}
\instructions{The WPs represent the main phase of the project. Their number is
  limited, usually about 4 to 8 in NPRP projects. It is recommended to have one WP
  for project management and one for communication and exploitation of the
  results. If the project involves research with human subjects or research with
  vertebrate animals the work must be described by jurisdiction and for all the
  institutions engaged in research as defined by the Ministry of Public Health
  [see \url{https://www.moph.gov.qa/about-us/Pages/research.aspx}].
  Please note that obtaining institutional ethical committee(s)
  review and approval is a regulatory requirement not a deliverable.  

  If relevant, provide a diagram showing the organization of the WPs among
  themselves (technical flow chart).

  For each WP describe:
  \begin{itemize}
  \item the objectives if any,
  \item the responsible person and the PIs involved,
  \item the detailed work program (a few lines per tasks, etc.),
  \item the deliverables (tangible outcomes produced by the WPs or by the tasks),
  \item the contributions of the main participants (Who does what?, PIs, post-docs, PhD/graduate students),
  \item a description of the methods and technical choices and the way in which solutions will be achieved,
  \item the risks and the back-up solutions envisaged,
  \item the performance site.
  \end{itemize}
}

\subsubsection{\underline{WP1:} WP title (optional)}

\begin{center}
% Total width must be 1
\begin{tabular}{|P{0.32}|P{0.136}|P{0.136}|P{0.136}|P{0.136}|P{0.136}|}
\hline
\textbf{Work package number} &
  \multicolumn{5}{l|}{
    \begin{tabular}{P{0.05}|P{0.17}|P{0.12}|P{0.17}|P{0.12}}
      1 & \textbf{Start month} & 0 & \textbf{End month} & 0
    \end{tabular}} \\
\hline
\textbf{Work package title} & \multicolumn{5}{l|}{} \\
\hline
\textbf{Name of participant} &  &  &  &  &  \\
\hline
\textbf{Effort days per participant} & 0 & 0 & 0 & 0 & 0 \\
\hline
\textbf{Performance site (if applicable)} &  &  &  &  &  \\
\hline
\multicolumn{6}{|P{1}|}{
%
  \textbf{Objectives of this work package:}

  \medskip

%
} \\
\hline
\multicolumn{6}{|P{1}|}{
%
  \textbf{Description of work:}
  \instructions{(where appropriate, broken down into tasks), WP leader and role of the main Participants}

  \medskip

  \underline{Task 1.1:}

  \underline{Task 1.2:} 
%
} \\
\hline
\multicolumn{6}{|P{1}|}{
%
  \textbf{Deliverables:}
  \instructions{(brief description and month of delivery)}

%
} \\
\hline
\end{tabular}
\end{center}


\newpage
\subsubsection{\underline{WP2:} WP title (optional)}
%
\begin{center}
% Total width must be 1
\begin{tabular}{|P{0.32}|P{0.136}|P{0.136}|P{0.136}|P{0.136}|P{0.136}|}
\hline
\textbf{Work package number} &
  \multicolumn{5}{l|}{
    \begin{tabular}{P{0.05}|P{0.17}|P{0.12}|P{0.17}|P{0.12}}
      2 & \textbf{Start month} & 0 & \textbf{End month} & 0
    \end{tabular}} \\
\hline
\textbf{Work package title} & \multicolumn{5}{l|}{} \\
\hline
\textbf{Name of participant} &  &  &  &  &  \\
\hline
\textbf{Effort days per participant} & 0 & 0 & 0 & 0 & 0 \\
\hline
\textbf{Performance site (if applicable)} &  &  &  &  &  \\
\hline
\multicolumn{6}{|P{1}|}{
%
  \textbf{Objectives of this work package:}

  \medskip

%
} \\
\hline
\multicolumn{6}{|P{1}|}{
%
  \textbf{Description of work:}
  \instructions{(where appropriate, broken down into tasks), WP leader and role of the main Participants}

  \underline{Task 2.1:}

  \underline{Task 2.2:}
%
} \\
\hline
\multicolumn{6}{|P{1}|}{
%
  \textbf{Deliverables:}
  \instructions{(brief description and month of delivery)}
%
} \\
\hline
\end{tabular}
\end{center}


\subsection{WPs SCHEDULE, DELIVERABLES AND MILESTONES}
\label{sec:method:deliv}
\instructions{Present in graphic form (Gantt diagram, for example) a schedule of
  the WPs with their main tasks.  

  Provide a table summarizing all the project
  deliverables (task number, date, title, person \& institution responsible).
  Evaluators should be able to judge the effort in each WP. If awarded, the list
  of deliverables will be used to monitor the project progress. \textbf{Therefore,
  only a reasonable number of main deliverables should be provided.} Describe the
  main scientific and/or technical milestones.}

\begin{center}
%\begin{table}[h]
\begin{tabular}{|C{0.07}|P{0.27}|C{0.15}|C{0.15}|C{0.15}|C{0.1}|C{0.1}|}
\hline
  \rowcolor{headercolor}
  \textbf{WP\#} &
  \multicolumn{1}{c|}{\textbf{WP title}} &
  \textbf{Name of the responsible person} &
  \textbf{Person days inside Qatar} &
  \textbf{Person days outside Qatar} &
  \textbf{Start month} &
  \textbf{End month} \\
\hline
  1  &  &  &  &  &  &  \\
\hline
  2  &  &  &  &  &  &  \\
\hline
     &  &  & \textbf{Total: X}  & \textbf{Total: Y} & & \\
\hline
\end{tabular}
%\caption{List of work packages}
%\end{table}
\end{center}


\begin{center}
%\begin{table}[h]
\begin{tabular}{|C{0.15}|P{0.38}|C{0.17}|P{0.2}|C{0.10}|}
\hline
  \rowcolor{headercolor}
  \textbf{Deliverable\#} &
  \multicolumn{1}{c|}{\textbf{Deliverable title}} &
  \textbf{Name of the responsible person} &
  \textbf{Type of deliverable} &
  \textbf{Delivery month} \\
\hline
   &  &  & \instructions{Document, report, plan designs, prototype, software, website, patent filing, media action, publications, etc.} &  \\
\hline
\end{tabular}
%\caption{List of deliverables}
%\end{table}
\end{center}


\newpage
TIMETABLE

\instructions{\LaTeX\ template note: Complete the timetable in file \texttt{timeline.tex}, generate the
pdf of that one page, and replace this page by it on the final pdf.}

\newpage
\thispagestyle{fancy}

\section{EXPECTED IMPACT OF THE PROJECT}
\label{sec:impact}
\instructions{Use quantitative indicators and targets, whenever possible.}

\subsection{OUTPUTS AND OUTCOMES}
\label{sec:impact:out}
\instructions{Describe the specific proposed outputs (examples are provided in
  the Definitions and Terminologies document --
  \url{https://www.qnrf.org/en-us/Funding/Research-Programs/National-Priorities-Research-Program-NPRP})
  and expected outcomes (peer-reviewed publications, patents, creative works, and
  others) of this research and indicate which ones you plan to create within the
  scope of the project.  Avoid generic descriptions and be as specific as
  possible.}

\begin{adjustwidth}{0.2in}{0in}
\end{adjustwidth}


\subsection{ALIGNMENT TO NPRP-S PRIORITY THEMES}
\label{sec:impact:align}
\instructions{Provide a strong rationale justifying the alignment of your
  proposal with NPRP-Standard priority theme(s).}

\begin{adjustwidth}{0.2in}{0in}
\end{adjustwidth}

\subsection{SOCIAL, HEALTH, ECONOMIC, AND ENVIRONMENTAL IMPACT}
\label{sec:impact:social}
\instructions{Awarded research projects are expected to focus on Qatar's
  development needs and to accelerate research in national fields of excellence.
  Describe how your project will contribute to:
  \begin{itemize}
  \item improving innovation capacity and the integration of new knowledge in Qatar and/or the region,
  \item the impact on any other environmental, health, economic and socially important aspects,
  \item improving research and education capacities in Qatar.
  \end{itemize}
  Describe specific future plans and potential research developments after the
  project's completion, if applicable. Avoid generic descriptions that could apply
  to any project.}

\begin{adjustwidth}{0.2in}{0in}
\end{adjustwidth}


\subsection{COMMUNICATION AND EXPLOITATION OF RESULTS}
\label{sec:impact:comm}
\instructions{Provide a specific plan for the communication and exploitation of
  the project's results as well as the promotion of the project and its findings.
  Examples could include workshops, public seminars, professional training, policy
  papers, media outreach. Avoid generic descriptions that could apply to any
  project.}

\begin{adjustwidth}{0.2in}{0in}
\end{adjustwidth}

\section{RISK ASSESSMENT AND MITIGATION}
\label{sec:risk}
\instructions{Assess the risks associated with conducting or completing the
  research project as proposed, and outline plans for mitigating them. Risks
  include threats to health and safety to researchers and subjects of research,
  delays, for example, in recruitment (researchers, students, patients, etc.) or
  procurement, or in obtaining regulatory (e.g., ethical) permissions. Risks
  could also be scientific, such as degrees of certainty, long execution times
  of codes or malfunction of equipment and facilities. You should describe the
  nature and degree of risk, its likelihood of occurrence, and outline plans of
  mitigating it.
}

\begin{adjustwidth}{0.2in}{0in}
\end{adjustwidth}

\section{RESEARCH TEAM DESCRIPTION}
\label{sec:team}

\subsection{DESCRIPTION, SUITABILITY AND COMPLEMENTARITY OF THE RESEARCH TEAM
MEMBERS}
\label{sec:team:memb}
\instructions{Briefly describe each PI, the research team and their
  institutional affiliation. Provide an assessment of their qualification to
  participate in the project and their specific roles in the project.
  Qualifications can include past achievements, research outcome indicators
  (publications, patents), and the specific expertise the research team member
  adds to the project, etc.  

  Show how team members complement each other's
  expertise and describe the added value of the collaborations, with a special
  emphasis on the contribution by the LPI and his/her team.  Interdisciplinarity
  and the inclusion of specific Qatar-based and external research teams must be
  justified in accordance with the project objectives. Outline how international
  collaborations (if applicable to your project) will enhance the research
  capacity and scientific excellence in Qatar.}

\begin{adjustwidth}{0.2in}{0in}
\end{adjustwidth}

\subsubsection{LPI}
\instructions{Provide justification of the LPI's capability to coordinate the
  project and the research team(s).}

\begin{adjustwidth}{0.2in}{0in}
\end{adjustwidth}

\subsubsection{PIs}

\begin{adjustwidth}{0.2in}{0in}
\end{adjustwidth}

\subsubsection{Other research staff (RAs, PDFs, students, etc, ...)}

\begin{adjustwidth}{0.2in}{0in}
\end{adjustwidth}

\subsection{RELEVANT PUBLICATIONS}
\label{sec:team:pub}
\instructions{List up to five publications or patents of the research team most
  relevant to the proposed project.}

\begin{adjustwidth}{0.2in}{0in}
\noindent
\emph{The Lord of the Rings.} J. R. R. Tolkien.
Journal of the Middle-earth, Volume XXX, Issue X, 1954, Pages 1-1000.

\medskip

\noindent
\emph{The Hitchhiker's Guide to the Galaxy.}
Ford Prefect and Arthur Dent. Technical Report, 4200.

\end{adjustwidth}

\subsection{CONSULTANTS/SERVICE PROVIDERS INVOLVED IN THE PROJECT}
\label{sec:team:cons}
\instructions{Does the participant plan to outsource certain tasks (please note
  that core tasks with research and intellectual contribution of the project must
  not be outsourced and carried out by consultants -- see definition of a
  consultant at
  \url{https://www.qnrf.org/en-us/Funding/Research-Programs/National-Priorities-Research-Program-NPRP})?

  If yes, please describe and justify the tasks to be outsourced.  List the
  consultants and service providers to be involved in the project and describe
  their contribution.}

\begin{adjustwidth}{0.2in}{0in}
\end{adjustwidth}


\subsection[INVOLVEMENT OF RESEARCH END-USERS IN THE PROJECT]{Involvement of
Research End-Users in the Project}
\label{sec:team:enduser}
\instructions{If the project team involves a research end-user(s) (public
  organizations, companies, non-profit organizations) that is contributing to the
  project as a collaborative institution, then describe its contribution. Check
  the QNRF's co-funding policy.  Specify in which manner the research end-user(s)
  may benefit from the project outcomes and how, specifically, it will support the
  project. Specify for each activity whether the support is a cash and/or in- kind
  contribution. If the co-funder is not a collaborative institution, provide a
  short presentation of the co- funder and specify how the co-funder will fund the
  project.
  \begin{itemize}
  \item Cash contribution: describe and justify the cash contribution of the
  research end-user(s) that plans to participate in the project. State the value
  of the cash contribution. Specify who in the project team will benefit from the
  cash contribution (overall team or some collaborative institutions only).
  \item In-kind contribution: describe and justify the in-kind contribution of the
  research end-user(s) that plans to participate in the project. Estimate the
  value of the in-kind contribution.
  \end{itemize}
}

\begin{adjustwidth}{0.2in}{0in}
\end{adjustwidth}

%% cash contribution table: TODO

%% in-kind contribution table: TODO

\section{RENEWAL JUSTIFICATION (if applicable)}
\instructions{If this proposal is a renewal application, LPIs need to justify
  the request by describing the excellent outcomes/ accomplishments that have
  resulted from the previous NPRP-award (including publications, patent(s),
  human resources development at the postdoctoral, graduate and undergraduate
  levels, etc.) and how these outcomes/accomplishments support the current
  proposal.}

\begin{adjustwidth}{0.2in}{0in}
\end{adjustwidth}

\section{\refname}
\instructions{Include the list of bibliographic references used in the research plan.}

\bibliographystyle{plain}
\bibliography{researchPlan}

\end{document}
